\section{Methods} \label{sec:methods}

For initial tests of proposed, we used a simple genome model where sites had explicitly ascribed effects.
This allowed us to design trials with known effects in genomes and then assess the capability assays to recover this information.
Effects were assigned randomly, to genome sites and multiple effects were allowed to overlap on the same site.
When evaluating the knockout effects on a genome, we summed constituent effects and defined a deleterious outcome if these effects summed to a value greater than 1.0.
Additive sites contributed a constant amount TODOAMOUNT to the sum if knocked out.
Epistatic sites are created in sets of TODOAMOUNT sites, and an effect of TODO was only accured if all sites in a set were knocked out.

This model has obvious limitations that mean that real-applications would be more challenging and does not reflect those challenges, the most obvious one being that the results are deterministic.
The next steps for this project is to try with test models incorporating stochastic effects, with pre-existing alife systems that are tractable to know the underlying full structure of genomes, and with heavier, more opaque systems that this methodology is ultimately designed to help with.

The following section reports the design of proposed assays and the results of a validation test for each.
Software and data for these initial experiments are available via GitHub at \url{https://github.com/mmore500/cryptic-sequence-concept}.
